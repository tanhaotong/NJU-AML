\documentclass[a4paper,UTF8]{article}
\usepackage{ctex}
\input{note-setup-leftsidebox}
\title{hw2}
\author{谭昊童  \href{mailto:231300039@smail.nju.edu.cn}{231300039@smail.nju.edu.cn}}

\begin{document}
\maketitle
\section{(25 points) 逆归结核心规则应用  }


逆归结是一种从已知事实(证据)和背景知识反向推导出有效假设的逻辑推理方法。在逆归结中,核心操作包括四种规则:吸收(Absorption)、辨识(Identification)、内构(Intra-construction)和互构(Inter-construction)(提示:详细规则可参考机器学习(西瓜书)第361页)。现在考虑如下情境,Tom报名参加了市级举重大赛。与比赛结果相关的部分规则和事实如下:  
\begin{itemize}
    \item R1:若一个人“强壮”且“有技巧”,则他会“获胜”(记为:获胜 $\leftarrow$ 强壮 $\wedge$ 有技巧); \item R2:若一个人“坚持锻炼”,则他会“强壮”(记为:强壮 $\leftarrow$ 坚持锻炼); \item R3:若一个人“强壮”且“对手弱”,则他会“获胜”(记为:获胜 $\leftarrow$ 强壮 $\wedge$ 对手弱); \item R4:一个人不能同时“有技巧”和“运气好”(记为: $\neg$ 有技巧 $\vee$ $\neg$ 运气好);  
\end{itemize}


已知事实:Tom 最终在举重大赛中获胜了。  

1. 假如大赛前 Tom 因意外受伤, 没有坚持锻炼。结合上述背景知识, 你能否判断 Tom 是否会取得胜利? 请详细说明推理过程及依据。(8 分)

2. 有观众猜测, Tom 获胜是因为存在一条未公开的 “神秘规则” (形式为 获胜 $\leftarrow$ 强壮 $\wedge$ X, X 是某个未提及的条件)。结合背景知识, 解释为什么 “X是运气好” 不符合逻辑约束。(8 分)

3. 教练复盘比赛时, 仔细观察 R1 (获胜 $\leftarrow$ 强壮 $\wedge$ 有技巧) 和 R3 (获胜 $\leftarrow$ 强壮 $\wedge$ 对手弱) 后提出: 两条规则中 “获胜” 的核心前提 “强壮” 是基础, 但真正决定胜负的是一种未被明确的 “神秘制胜条件” (记为 S)。该条件是 “有技巧” 和 “对手弱” 的共性本质, 且与 “强壮” 存在紧密逻辑关联。请明确 S 与 “强壮” “有技巧” “对手弱” 之间的逻辑关系(9 分) 

\begin{solution*}
\begin{enumerate}
    \item 不能。已知$\neg$坚持锻炼,但是仅由R2我们无法推出Tom是否强壮,而且缺少关于“有技巧”和“对手弱”的信息,因此无法确定Tom是否获胜。
    \item 假设X是运气好,则神秘规则为获胜 $\leftarrow$ 强壮 $\wedge$ 运气好。已知获胜 $\leftarrow$ 强壮 $\wedge$ 有技巧,由辨识规则可以得到\[\text{有技巧}\leftarrow\text{运气好}~~\text{获胜}\leftarrow\text{强壮}\wedge \text{有技巧}\]
    
    但是R4可以转化为\[\neg\text{有技巧}\leftarrow\text{运气好}\]

    矛盾,因此X不能是运气好。
    \item 由内构规则可以得到\[S\leftarrow\text{有技巧}~~\text{获胜}\leftarrow\text{强壮}\wedge S~~S\leftarrow\text{对手弱}\]
    
    关系:S是“有技巧”和“对手弱”的共性本质,“有技巧”和“对手弱”都能推出S.S和强壮构成获胜的充分条件。
\end{enumerate}
\end{solution*}
\section{(15 points) FOIL 算法信息增益计算应用题  }
小明正在开发一套家族关系智能识别系统,核心需求是让系统自动归纳“祖父(Grandfather, x, y)”的判定规则——即通过已知的“父亲”“男性”等基础关系,让系统自主学习“x是y的祖父”的逻辑条件。为实现这一目标,小明收集了相关数据并设计了候选规则,现需使用FOIL算法(一阶归纳学习算法)通过计算信息增益筛选最优规则。目标概念为需归纳的谓词Grandfather(x, y),语义为“x是y的祖父”;系统已知基础关系谓词包括Father(a, b)(表示“a是b的父亲”)和Male(a)(表示“a是男性”)。

小明标注的训练样本中,正例(即Grandfather(x, y) = True,x确实是y的祖父)共3个,分别为$P_{1}$(Peter,Tom)、$P_{2}$(John,Lily)、$P_{3}$(Mike, Jack);反例(即Grandfather(x, y) = False,x不是y的祖父)共3个,分别为$N_{1}$(Peter,Bob)、$N_{2}$(John,Jack)、$N_{3}$(Mike, Lily)。

一阶规则学习的初始规则为“最一般规则”(无任何前提条件):Grandfather(x, y) ← True,该规则会覆盖所有训练样本,因此初始正例覆盖数$p_{0}=3$,初始反例覆盖数$n_{0}=3$。  

父亲关系(Father):  

Father(Peter, David), Father(David, Tom), Father(David, Ann), Father(John, Paul), Father(Paul, Lily), Father(Paul, Bob), Father(Mike, Steve), Father(Steve, Jack), Father(Steve, Lucy)  



性别关系(Male):  

Male(Peter), Male(David), Male(John), Male(Paul), Male(Mike), Male(Steve)  

小明基于背景谓词设计了4条候选规则,具体如下:

1. 规则A: Grandfather$(x,y) \leftarrow$ Father$(x,z)$ 

2. 规则B: Grandfather$(x,y) \leftarrow$ Father$(z,y)$ 

3. 规则C: Grandfather$(x,y) \leftarrow$ Father$(x,z) \land$ Father$(z,y)$ 

4. 规则D: Grandfather$(x,y) \leftarrow$ Male$(x) \land$ Father$(x,z) \land$ Father$(z,y)$  

FOIL 算法通过信息增益衡量候选规则对“区分正例、排除反例”的贡献,信息增益越大,规则性能越 优,计算公式为:  

$$
\mathrm { G a i n } = p _ { 1 } \times \log _ { 2 } \left( \frac { p _ { 1 } } { p _ { 1 } + n _ { 1 } } \right) - p _ { 0 } \times \log _ { 2 } \left( \frac { p _ { 0 } } { p _ { 0 } + n _ { 0 } } \right)
$$  

其中:$p_{0}, n_{0}$ 为初始规则的正例覆盖数、反例覆盖数(已知 $p_{0}=3, n_{0}=3$);$p_{1}, n_{1}$ 为候选规则的正例覆盖数、反例覆盖数。  

1. 结合辅助事实,逐一推导规则A、B、C、D 的$p_{1}$(覆盖正例数)和$n_{1}$(覆盖反例数)分别计算4 条规则的FOIL 信息增益(8 分) 

2. 比较4 条规则的信息增益大小,指出信息增益最大的规则,并结合规则简洁性说明小明应选择的最 终最优规则及理由(7 分) 

\begin{solution*}
    \begin{table*}
\centering
\begin{tabular}{ccccc}
\hline
命题 & A规则值&B规则值&C规则值&D规则值\\
\hline
Grandfather(Peter,Tom)&True&True&True&True\\
Grandfather(John,Lily)&True&True&True&True\\
Grandfather(Mike, Jack)&True&True&True&True\\
\hline
Grandfather(Peter,Bob)&True&True&False&False\\
Grandfather(John,Jack)&True&True&False&False\\
Grandfather(Mike, Lily)&True&True&False&False\\
\hline
\end{tabular}
        \end{table*}
\begin{enumerate}
    \item\begin{enumerate}[(A)]
        \item $p_1=3,n_1=3,Gain=0$
        \item $p_1=3,n_1=3,Gain=0$
        \item $p_1=3,n_1=0,Gain=3\times \log_2(1)-3\times\log_2(\frac{3}{6})=3$
        \item $p_1=3,n_1=0,Gain=3\times \log_2(1)-3\times\log_2(\frac{3}{6})=3$
    \end{enumerate}
    \item 规则C和规则D信息增益最大,均为3。规则C较D来说少一条对性别的约束,更为简洁。虽然在现实生活中male是一个grandfather的合理约束,但在本题的样本中属于冗余信息。所以应该选择C作为最优规则。
\end{enumerate}
\end{solution*}

\section{(20 points) 马尔可夫决策过程与 Bellman 方程}
给定一个 MDP 四元组 $E=\langle\mathcal{S},\mathcal{A},P,R\rangle$, 其中 $\gamma\in[0,1)$ 为折扣因子, $R_{s\rightarrow s'}^{a}$ 表示在状态 $s$ 执行动作 $a$ 转移到 $s'$ 时的期望奖赏。  

1. 请写出最优状态值函数 $V^{*}(s)$ 和最优状态-动作值函数 $Q^{*}(s,a)$ 之间的相互转换关系式(即最优 Bellman 方程)。(10 分)

2. 策略改进定理指出,如果我们将策略选择的动作改变为当前 $Q$ 值最大的动作,策略会得到提升。设新策略为 $\pi'(s)=\arg \max_{a} Q^{\pi}(s,a)$,请证明对于任意状态 $s$,有 $V^{\pi'}(s) \geq V^{\pi}(s)$。(10 分)  

\begin{solution*}
    \begin{enumerate}
        \item $V^*(s)$是最优状态值函数,即\[V^*(s) = \max_{a \in \mathcal{A}} Q^*(s, a)\]
        
        最优状态-动作值函数$Q^*(s,a)$是执行动作a后的最优期望回报,即$$Q^*(s, a) = \sum_{s' \in \mathcal{S}} P(s' | s, a) \left[ R_{s\rightarrow s'}^{a} + \gamma V^*(s') \right]$$

        两式结合可得\[V^*(s) = \max_{a \in \mathcal{A}} \sum_{s' \in \mathcal{S}} P(s' | s, a) \left[ R_{s\rightarrow s'}^{a} + \gamma V^*(s') \right]\]
        \item 根据新策略的定义 $\pi'(s)=\arg \max_{a} Q^{\pi}(s,a)$,对于任意状态 $s$,我们可以得到:
        \begin{equation}
            \label{1}
        Q^{\pi}(s, \pi'(s)) = \max_{a} Q^{\pi}(s, a) \geq Q^{\pi}(s, \pi(s)) = V^{\pi}(s)
        \end{equation}
        利用Q值的定义展开式 \ref{1} 可得
        \begin{align*}
V^{\pi}(s) &\leq Q^{\pi}(s, \pi'(s)) \\
&= \mathbb{E}_{\pi'} \left[ R_{t+1} + \gamma V^{\pi}(S_{t+1}) \mid S_t = s \right]
\end{align*}
这里 $Q^{\pi}$ 是基于旧策略 $\pi$ 的价值函数,但动作是由新策略 $\pi'(s)$ 选择的.

在上面的期望中,出现了一项 $V^{\pi}(S_{t+1})$。我们可以对这一项再次应用 (式\ref{1}),即 $V^{\pi}(S_{t+1}) \leq Q^{\pi}(S_{t+1}, \pi'(S_{t+1}))$。代入上式:$$\begin{aligned}
V^{\pi}(s) &\leq \mathbb{E}_{\pi'} \left[ R_{t+1} + \gamma V^{\pi}(S_{t+1}) \mid S_t = s \right] \\
&\leq \mathbb{E}_{\pi'} \left[ R_{t+1} + \gamma Q^{\pi}(S_{t+1}, \pi'(S_{t+1})) \mid S_t = s \right] \\
&\text{(再次展开 Q 值)} \\
&= \mathbb{E}_{\pi'} \left[ R_{t+1} + \gamma \left( R_{t+2} + \gamma V^{\pi}(S_{t+2}) \right) \mid S_t = s \right] \\
&= \mathbb{E}_{\pi'} \left[ R_{t+1} + \gamma R_{t+2} + \gamma^2 V^{\pi}(S_{t+2}) \mid S_t = s \right]\\
&\leq \mathbb{E}_{\pi'} \left[ \sum_{i=0}^{k-1} \gamma^i R_{t+i+1} + \gamma^k V^{\pi}(S_{t+k}) \mid S_t = s \right]
\end{aligned}$$

当 $k \rightarrow \infty$ 时,由于 $\gamma \in [0, 1)$,衰减项 $\lim_{k \to \infty} \gamma^k V^{\pi}(S_{t+k}) = 0$(假设奖励是有界的)。剩下的求和项正是新策略 $\pi'$ 下的状态值函数 $V^{\pi'}(s)$ 的定义:$$\begin{aligned}
V^{\pi}(s) &\leq \lim_{k \to \infty} \mathbb{E}_{\pi'} \left[ \sum_{i=0}^{k-1} \gamma^i R_{t+i+1} \mid S_t = s \right] \\
&= V^{\pi'}(s)
\end{aligned}$$

所以对于任意状态 $s$,都有 $V^{\pi'}(s) \geq V^{\pi}(s)$。
    \end{enumerate}
\end{solution*}

\section{(20 points) 蒙特卡洛方法与重要性采样}  

1. 在蒙特卡洛强化学习中,请简述同策略(On-policy)和异策略(Off-policy)的主要区别。(5 分)  

2. 在异策略学习中, 我们需要估计目标策略 $\pi$ 下的期望回报, 但数据是根据行为策略 $b$ 采样的。请写出利用重要性采样 (Importance Sampling) 比率 $\rho_{t:T-1}$ 来修正回报 $G_t$ 的公式, 并解释为什么需要这样做。(10 分)  

3. 普通的重要性采样虽然是无偏的, 但存在什么主要问题? (5 分)  
\begin{solution*}
    \begin{enumerate}
        \item \begin{itemize}
            \item 同策略:用于生成采样数据的策略(行为策略)与我们要评估或改进的策略(目标策略)是同一个策略。
            \item 异策略:用于生成采样数据的策略(行为策略 $b$)与我们要评估或改进的策略(目标策略 $\pi$)是不同的。
        \end{itemize}
        \item$$\rho_{t:T-1} = \prod_{k=t}^{T-1} \frac{\pi(A_k | S_k)}{b(A_k | S_k)}$$
        $\pi(A_k | S_k)$指$\pi$策略在状态$S_k$下选择动作$A_k$的概率,$b(A_k | S_k)$指$b$策略在状态$S_k$下选择动作$A_k$的概率。
        
        利用重要性采样比率修正回报的公式为:$$\hat{G}_t = \rho_{t:T-1} G_t$$

        原因:我们的目标是计算回报 $G_t$ 在目标策略 $\pi$ 下的期望值 $\mathbb{E}_{\pi}[G_t]$。然而,我们手头的数据(轨迹)是根据行为策略 $b$ 采样生成的,实际上我们是在计算 $\mathbb{E}_{b}[G_t]$。

        设一条轨迹为 $\tau = \{S_t, A_t, S_{t+1}, A_{t+1}, \dots, S_T\}$。$P(\tau | \pi)$ 表示在策略 $\pi$ 下这条轨迹发生的概率。那么期望的定义公式为(这里用求和符号 $\sum$ 代表离散情况下的积分):$$\mathbb{E}_{\pi}[G_t] = \sum_{\tau} P(\tau | \pi) G_t(\tau)$$

        \begin{align*}
\mathbb{E}_{\pi}[G_t] &= \sum_{\tau} P(\tau | \pi) G_t(\tau) \\
&= \sum_{\tau} P(\tau | \pi) \cdot \frac{P(\tau | b)}{P(\tau | b)} \cdot G_t(\tau) \\
&= \sum_{\tau} P(\tau | b) \cdot\left( \frac{P(\tau | \pi)}{P(\tau | b)} \right)\cdot G_t(\tau)\\
&= \mathbb{E}_{b} \left[ \frac{P(\tau | \pi)}{P(\tau | b)} G_t \right]
\end{align*}
因为
\begin{align*}
P(\tau | \pi) &= \pi(A_t|S_t) \cdot p(S_{t+1}|S_t, A_t) \cdot \pi(A_{t+1}|S_{t+1}) \cdot p(S_{t+2}|S_{t+1}, A_{t+1}) \dots \\
P(\tau | b) &= b(A_t|S_t) \cdot p(S_{t+1}|S_t, A_t) \cdot b(A_{t+1}|S_{t+1}) \cdot p(S_{t+2}|S_{t+1}, A_{t+1}) \dots
\end{align*}
所以$$\frac{P(\tau | \pi)}{P(\tau | b)} = \frac{\prod_{k=t}^{T-1} \pi(A_k|S_k)}{\prod_{k=t}^{T-1} b(A_k|S_k)} = \prod_{k=t}^{T-1} \frac{\pi(A_k|S_k)}{b(A_k|S_k)}=\rho_{t:T-1}$$
所以乘上系数$\rho_{t:T-1}$即可修正回报。
        \item 方差极大,估计不稳定。
        
        原因:重要性采样比率 $\rho_{t:T-1}$ 是多个概率比值的连乘积($\prod \frac{\pi}{b}$)。当轨迹很长时,只要分母 $b(A_k|S_k)$ 中出现很小的值(即行为策略探索到了目标策略认为大概率的动作,但行为策略自身产生该动作的概率很低),或者分子远大于分母,这样的事情出现几次,连乘的结果就会呈指数级爆炸;反之则会接近于0。
    \end{enumerate}
\end{solution*}
\section{(20 points)时序差分学习}
Sarsa 与 Q-learning 时序差分学习结合了动态规划和蒙特卡洛方法的优点。  

1. 请分别写出Sarsa算法和Q-learning算法的单步更新公式。(8分)  

2. 基于上述公式, 说明为什么 Q-learning 是异策略 (Off-policy) 算法, 而 Sarsa 是同策略 (On-policy) 算法?

3. 在悬崖行走 (Cliff Walking) 等具有惩罚区域的环境中, Q-learning 和 Sarsa 学到的路径通常有何不同? 为什么? (5 分)
\begin{solution*}
    \begin{enumerate}
        \item 设 $S_t$ 为当前状态,$A_t$ 为当前动作,$R_{t+1}$ 为获得的奖励,$S_{t+1}$ 为下一状态,$\alpha$ 为学习率,$\gamma$ 为折扣因子。
        
        Sarsa 更新公式:$$Q(S_t, A_t) \leftarrow Q(S_t, A_t) + \alpha \left[ R_{t+1} + \gamma Q(S_{t+1}, A_{t+1}) - Q(S_t, A_t) \right]$$
        
        Q-learning 更新公式:$$Q(S_t, A_t) \leftarrow Q(S_t, A_t) + \alpha \left[ R_{t+1} + \gamma \max_{a} Q(S_{t+1}, a) - Q(S_t, A_t) \right]$$
        \item 在 Sarsa 的公式中,更新 $Q(S_t, A_t)$ 时用到的 TD 目标是 $R_{t+1} + \gamma Q(S_{t+1}, \mathbf{A_{t+1}})$。这里的 $A_{t+1}$ 是智能体在下一时刻实际将会执行的动作。它是根据当前的行为策略(例如 $\epsilon$-greedy)采样出来的。这意味着,算法在评估的策略,正是智能体当前用来行动的策略。学的就是正在做的,两者一致,所以是同策略。
        
        在 Q-learning 的公式中,更新 $Q(S_t, A_t)$ 时用到的 TD 目标是 $R_{t+1} + \gamma \mathbf{\max_{a} Q(S_{t+1}, a)}$。这里它假设下一时刻会执行最优动作(Greedy Action,即 Q 值最大的动作),而不管智能体实际上下一时刻会采取什么动作。这意味着,算法在评估的策略,是基于当前 Q 值的最优策略,而不是智能体当前实际用来行动的策略。学的和正在做的不一样,所以是异策略。
        \item Q-learning:通常会学到一条最优路径(Shortest Path),即贴着悬崖边缘走。这条路最近,步数最少。因为它在更新时使用的是 $\max Q$,它默认自己未来会完全按照最优策略行事(不会犯错)。只要不掉下去,边缘的路就是分数最高的。
        
        Sarsa:通常会学到一条安全路径(Safe Path),即远离悬崖边缘,稍微绕远一点的路。因为它在更新时使用的是实际动作 $A_{t+1}$。如果走在悬崖边缘,由于 $\epsilon$-greedy 的存在,$A_{t+1}$ 有一定概率($\epsilon$)是“跳入悬崖”。这个负回报会被计入当前边缘状态的价值中,从而降低边缘状态的 Q 值,最终学会选择一条远离危险的安全路线。
    \end{enumerate}
\end{solution*}
\end{document}