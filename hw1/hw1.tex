\documentclass[a4paper,UTF8]{article}
\usepackage{ctex}
\input{note-setup-leftsidebox}
\title{hw1}
\author{谭昊童  \href{mailto:231300039@smail.nju.edu.cn}{231300039@smail.nju.edu.cn}}
\def \R {\mathbb{R}}
\def \L {\mathcal{L}}
\def \w {\mathbf{w}}
\def \x {\mathbf{x}}
\begin{document}
\maketitle
\section{(20 points) 聚类}
\begin{enumerate}
    \item 谈谈对聚类的理解,包括其应用场景和难点。(5分)
\item 针对西瓜数据集3.0(表1),使用密度和含糖率这两个属性,计算样本1和样本2之间的闵可夫斯基
距离, 以及样本1和样本3之间的闵可夫斯基距离.本题中,闵可夫斯基距离(Minkowskidistance)
的参数p取值为1, 2, 3.观察计算结果, 在不同的 p 的取值下, 这三个样本的相似性关系是否一
致?(5 分)
\item 针对表1,计算属性“色泽”上三个离散值“青绿”,”乌黑”和“浅白”两两之间的VDM(Value Difference Metric)
距离,VDM 距离中参数p取值为1,类别按照数据集的真实类别判定.验证VDM距离满足直递性.
(5 分)
\item 按照自己的理解介绍k均值聚类算法,并分析其适用场景与存在的不足。(5分)
\end{enumerate}

\begin{solution*}
    \begin{enumerate}
        \item 聚类是一种\textbf{无监督学习}任务。聚类算法的目标是将一个数据集中没有标签的样本,按照它们内在的相似性或距离,自动地划分成若干个簇。
        
        理想的聚类结果应该满足两个特性:簇内相似性高;簇间相似性低。

        \textbf{应用场景:}图像分割、异常检测等。

        \textbf{难点:}距离度量的选择、簇数的选择、维度诅咒等。

        \item 
        
        \begin{table*}[htbp]
\centering
\begin{tabular}{ccc}
\hline
p& 1和2 & 1和3 \\
\hline
1& 0.1610 & 0.2590 \\
2& 0.1140 & 0.2059 \\
3& 0.1016 & 0.1981 \\
\hline
\end{tabular}
        \end{table*}

结论: 在不同的 p 取值下, 相似性关系(谁更近)保持一致。

        \item \begin{align*}
            VDM(\text{青绿,乌黑})&=|P(\text{好瓜}|\text{青绿})-P(\text{好瓜}|\text{乌黑})|^1+|P(\text{坏瓜}|\text{青绿})-P(\text{坏瓜}|\text{乌黑})|^1\\
            &=|\frac{3}{6}-\frac{4}{6}|+|\frac{3}{6}-\frac{2}{6}|\\
            &=\frac{1}{3} \\
            VDM(\text{青绿,浅白})&=|P(\text{好瓜}|\text{青绿})-P(\text{好瓜}|\text{浅白})|^1+|P(\text{坏瓜}|\text{青绿})-P(\text{坏瓜}|\text{浅白})|^1\\
            &=|\frac{3}{6}-\frac{1}{5}|+|\frac{3}{6}-\frac{4}{5}|\\
            &=\frac{3}{5} \\
            VDM(\text{乌黑,浅白})&=|P(\text{好瓜}|\text{乌黑})-P(\text{好瓜}|\text{浅白})|^1+|P(\text{坏瓜}|\text{乌黑})-P(\text{坏瓜}|\text{浅白})|^1\\
            &=|\frac{4}{6}-\frac{1}{5}|+|\frac{2}{6}-\frac{4}{5}|\\
            &=\frac{14}{15} \\
        \end{align*}

        验证直递性:
        \begin{align*}
            VDM(\text{青绿,乌黑})+VDM(\text{乌黑,浅白})&=\frac{1}{3}+\frac{14}{15}=\frac{19}{15}>\frac{3}{5}=VDM(\text{青绿,浅白})\\
            VDM(\text{青绿,浅白})+VDM(\text{乌黑,浅白})&=\frac{3}{5}+\frac{14}{15}=\frac{23}{15}>\frac{1}{3}=VDM(\text{青绿,乌黑})\\
            VDM(\text{青绿,乌黑})+VDM(\text{青绿,浅白})&=\frac{1}{3}+\frac{3}{5}=\frac{14}{15}\ge\frac{14}{15}=VDM(\text{乌黑,浅白})\\
        \end{align*}

        满足直递性。

        \item K-Means是最著名、最简单的聚类算法之一。它的核心思想是“最小化簇内误差平方和 (SSE)”,它试图找到 $K$ 个中心点(质心),并使得数据集中每个点到其所属簇的质心的距离平方和最小。
        
        K-Means 的迭代步骤如下:
        \begin{enumerate}[(1)]
            \item 初始化:首先,预先指定要分成的簇数 $ K $。然后,从数据集中随机选择 $K$ 个样本点作为初始的簇质心。
            \item 分配:对于数据集中的每一个样本点,分别计算它到 $K$ 个质心的距离,将该样本分配给距离它最近的那个质心所代表的簇。
            \item 簇质心更新:当所有样本都分配完毕后,取该簇中所有样本点的均值作为新的质心。
            \item 迭代:重复执行第 2 步和第 3 步,直到满足某个停止条件。
        \end{enumerate}
        
        \textbf{适用场景:}
        \begin{itemize}
            \item 数据量大且维度较低: K-Means 算法复杂度相对较低,计算效率高,适合处理大规模数据集。
            \item 簇的形态为凸形(球状): K-Means 倾向于发现大小相似、形状近乎球形(或凸形)的簇,并且簇与簇之间有较好的分离度。
            \item 数值型数据: 算法依赖于计算“均值”和“欧氏距离”,因此它天然适用于连续的数值型特征。
            \item 快速得到初步结果: 作为一种简单高效的基准模型,K-Means 常常用于对数据进行快速的初步探索性分析。
        \end{itemize}
        \textbf{存在的不足}
        \begin{itemize}
            \item $ K $值需要预先指定: K-Means 无法自动确定最佳簇数$ K $,$ K $ 值的选择非常依赖经验或辅助方法。
            \item 对初始质心敏感: 随机选择初始质心可能导致算法陷入局部最优解,而不是全局最优解。不同的初始点可能导致完全不同的聚类结果。
            \item 对异常值(Outliers)敏感: 因为 K-Means 使用“均值”来更新质心,少数几个极端异常值就可能严重影响聚类结果。
            \item 无法处理非球状簇: 它无法很好地识别非凸形状的簇(如月牙形、环形)或密度差异很大的簇。
        \end{itemize}
        
    \end{enumerate}

\end{solution*}

\section{(20 points) 降维}  
\begin{enumerate}
    \item 谈谈对维度灾难的理解,包括其原因以及危害,还有可以从哪些方面缓解维度灾难。(5分)  

\item 本题考察 PCA 相关的线性代数基础知识以及基本操作。给定 $d$ 维空间中 $m$ 个样本构成的矩阵  

$$
X = [ x _ { 1 } ^{\top} ; \ldots ; x _ { m } ^{\top} ] \in \mathbb { R } ^{m \times d} ,
$$  

$\hat{X} \in \mathbb{R}^{m \times d}$ 为 $X$ 中心化后得到的矩阵。严格的协方差矩阵具有 $\frac{1}{m-1}$ 因子,由于常数对本题分析结果无影响,所以在本题的讨论中忽略该常数因子。  

\begin{enumerate}[(1)]
    \item $\hat{X}^{\top} \hat{X}$ 和 $\hat{X} \hat{X}^{\top}$ 为什么是半正定矩阵? 二者的特征值有什么联系? 受此启发, 请思考当特征维度远大于样本个数时 $(d \gg m)$, 使用特征值分解求解 PCA 应如何执行将更加高效? (8 分)  
    
    \item 针对以下样本矩阵 (包含 5 个示例, 每个示例 2 维), 请对其进行主成分分析, 将样本降至一维并写出详细计算过程。(7 分)  

\begin{align*}
    X ^{\top} = { \left( \begin{array}{lllll} 
        { 3 } & { 4 } & { 4 } & { 6 } & { 3 } \\
        { 2 } & { 3 } & { 2 } & { 3 } & { 0 } 
    \end{array} \right) }
\end{align*}
\end{enumerate}

\end{enumerate}
\begin{solution*}
    \begin{enumerate}
        \item 维度灾难是指在数据挖掘和机器学习中,当数据的特征维度(即 $d$)急剧增加时,一系列问题随之产生,导致算法性能急剧下降、计算成本激增的现象。
        
        根本原因:随着维度的增加,数据空间的体积会呈指数级增长(例如 $V \propto L^d$)。为了保持相同的数据密度,我们所需要的样本量 $m$ 也必须呈指数级增长。然而在现实中,样本量($m$)的增长速度远远跟不上维度($d$)的增长速度。
        
        主要危害:\begin{itemize}
            \item 数据稀疏性与近邻失效:在高维空间中,样本点会变得极其稀疏。任意两个样本点之间的距离都倾向于变得很大。这使得基于“邻域”的算法(如KNN)和基于密度的算法(如 DBSCAN)几乎失效,因为“局部”的概念失去了意义。
            \item 距离度量失去意义:在高维空间中,任意一点到其他所有点的“最近距离”和“最远距离”之间的相对差距会趋向于0,所有基于距离度量的算法(如 K-Means 聚类、KNN)的性能都会严重退化。
            \item 计算复杂度剧增:算法的计算和存储成本通常与维度 $d$ 相关(例如 $O(d)$, $O(d^2)$)。高维度会使算法变得极其缓慢,甚至在计算上不可行。
            \item 过拟合风险:当特征维度 $d$ 远远大于样本数 $m$ 时($d \gg m$),模型有太多的“自由度”。这使得模型很容易背下训练数据中的噪声,而不是学习到底层的数据规律,导致模型在训练集上表现很好,但在测试集上表现很差。
        \end{itemize}
        
        缓解方法:\begin{enumerate}
            \item 特征选择:通过选择最相关的特征子集,减少冗余和无关特征,从而降低维度。
            \item 特征提取/ 降维:通过将原始高维特征投影到一个新的低维空间中,生成全新的、数量更少的特征。
        \end{enumerate}
        
        \item\begin{enumerate}[(1)]
            \item \textbf{证明半正定矩阵:}设$v\in \R^m$为任意非零向量,则\begin{align*}
                v^{\top}(\hat{X}^{\top}\hat{X})v&=(v^{\top}\hat{X}^{\top})(\hat{X}v)\\
                &=(\hat{X}v)^{\top}(\hat{X}v)\\
                &=\sum_{i=1}^d(\hat{X}v)_i^2\ge0
            \end{align*}
            因此$\hat{X}^{\top}\hat{X}$是半正定矩阵。同理可证$\hat{X}\hat{X}^{\top}$也是半正定矩阵。

            \textbf{二者有完全相同的非零特征值。}设$\lambda\ne0$为$\hat{X}^{\top}\hat{X}$的特征值,$v$为任意非零向量,则有\[(\hat{X}^\top\hat{X})v=\lambda v\]

            两边左乘$\hat{X}$得\[\hat{X}(\hat{X}^\top\hat{X})v=\lambda \hat{X}v\]
            即\[(\hat{X}\hat{X}^\top)(\hat{X}v)=\lambda (\hat{X}v)\]
            因为$\lambda\ne0,v\ne 0$,所以$\hat{X}v\ne0$,设$u=\hat{X}v$,则$u\ne0$且\[(\hat{X}\hat{X}^\top)u=\lambda u\]
            因此$\lambda$也是$\hat{X}\hat{X}^\top$的特征值。

            \textbf{如果要对$d\times d$的协方差矩阵$\hat{X}^\top\hat{X}$进行特征值分解},先对$m\times m$的矩阵$\hat{X}\hat{X}^\top$进行特征值分解,得到非零特征值$\lambda_i$和对应的特征向量$u_i$.

            设$v_i=\hat{X}^\top u_i$,\[(\hat{X}^\top\hat{X})v_i=\hat{X}^\top(\hat{X}\hat{X^\top}u_i)=\hat{X}^\top\lambda_i u_i=\lambda_i v_i\],所以$v_i$是$\hat{X}^\top\hat{X}$的特征向量。

            \item 对X进行中心化,将 $X$ 的每一行减去均值 $ \mu^{\top} = (4, 2) $:$$\hat{X} = X - \mu = { \left( \begin{array}{ll} 3-4 & 2-2 \\ 4-4 & 3-2 \\ 4-4 & 2-2 \\ 6-4 & 3-2 \\ 3-4 & 0-2 \end{array} \right) } = { \left( \begin{array}{rr} -1 & 0 \\ 0 & 1 \\ 0 & 0 \\ 2 & 1 \\ -1 & -2 \end{array} \right) }$$
            
            计算\begin{align*}
                \hat{X}^\top\hat{X}&={ \left( \begin{array}{rrrrr} -1 & 0 & 0 & 2 & -1 \\ 0 & 1 & 0 & 1 & -2 \end{array} \right) }{ \left( \begin{array}{rr} -1 & 0 \\ 0 & 1 \\ 0 & 0 \\ 2 & 1 \\ -1 & -2 \end{array} \right) }= \left( \begin{array}{rr} 6 & 4 \\
                4&6 \end{array} \right)
                \end{align*}

                解方程$\det(\hat{X}^\top\hat{X}-\lambda I)=0$,得到特征值$\lambda_1=10,\lambda_2=2$.

                因为题目要求降为一维,所以计算最大特征值$\lambda_1=10$对应的特征向量$v_1$.

                解方程$(\hat{X}^\top\hat{X}-10 I)v_1=0$,\[\left(\begin{array}{rr} -4 & 4 \\ 4 & -4 \end{array}\right)\left(\begin{array}{r} v_{11} \\ v_{12} \end{array}\right)=-4\left(\begin{array}{r}
                v_{11}-v_{12}\\
                -v_{11}+v_{12}
                \end{array}\right)=0\]

                得到$v_{11}=v_{12}$,单位化后,$v_1=\left(\frac{\sqrt{2}}{2},\frac{\sqrt{2}}{2}\right)^\top$.

                将样本投影到主成分方向上,得到降维后的样本:\begin{align*}
                    Y&=\hat{X}v_1={ \left( \begin{array}{rr} -1 & 0 \\ 0 & 1 \\ 0 & 0 \\ 2 & 1 \\ -1 & -2 \end{array} \right) }{ \left( \begin{array}{r} \frac{\sqrt{2}}{2} \\ \frac{\sqrt{2}}{2} \end{array} \right) }={ \left( \begin{array}{r} -\frac{\sqrt{2}}{2} \\ \frac{\sqrt{2}}{2} \\ 0 \\ \frac{3\sqrt{2}}{2} \\ -\frac{3\sqrt{2}}{2} \end{array} \right) }
                \end{align*}
        \end{enumerate}
    \end{enumerate}
\end{solution*}

\section{(20 points) 特征选择 } 
\begin{enumerate}
    \item 简述特征选择的目的,以及一些常用的特征选择方法。(5分)  

\item 当样本特征很多, 而样本数相对较少时, 模型很容易陷入过拟合, 为了缓解过拟合问题, 可引入正则化项。请通过公式说明 L1 范数和 L2 范数在正则化中的用法, 以及分析为什么它们可以缓解过拟合问题, 过程中结合画图分析 L1 与 L2 范数的差异。(10 分)  

\item 字典学习与压缩感知都有对稀疏性的利用, 请你分析两者对稀疏性利用的异同点。(5 分)  
\end{enumerate}

\begin{solution*}
    \begin{enumerate}
        \item 目的:\begin{itemize}
            \item 缓解维度灾难与过拟合: 当特征过多而样本过少时,模型容易过拟合。移除不相关的特征(噪声)有助于提升模型的泛化能力。

\item 降低计算和存储成本: 更少的特征意味着模型训练更快、推理更快、占用存储空间更少。

\item 避免冗余信息: 移除高度相关的(冗余的)特征,使模型更简洁高效。
        \end{itemize}
        常用方法:\begin{itemize}
            \item 过滤法:基于统计指标(如方差、相关系数、卡方检验等)评估每个特征与目标变量的关系,选择排名靠前的特征。
            \item 包装法:使用特定的机器学习算法作为“黑盒”,通过搜索不同的特征子集来评估模型性能,选择性能最好的子集。
            \item 嵌入法:在模型训练过程中自动进行特征选择,如决策树中的特征重要性、Lasso 回归中的 L1 正则化等。
        \end{itemize}
        \item 假设我们有一个损失函数 $L(\mathbf{w})$.
        
        L1 正则化目标函数: $J_{L1}(\mathbf{w}) = L(\mathbf{w}) + \lambda \sum_{j=1}^{d} |w_j|$

        L2 正则化目标函数: $J_{L2}(\mathbf{w}) = L(\mathbf{w}) + \lambda \sum_{j=1}^{d} w_j^2$

        为什么可以缓解过拟合:当 $d \gg m$ 时,模型过于复杂,它有足够的能力去“记住”训练数据中的噪声。这通常体现在模型参数(权重)$\mathbf{w}$ 非常大。添加正则项后,最小化目标函数要求模型在拟合数据(最小化$L(\mathbf{w})$)的同时,也要保持权重的“规模”较小(最小化正则项)。这迫使模型学习到更简单、更平滑的函数,从而减少对训练数据噪声的敏感性,提升泛化能力。

        \begin{figure}[htbp]
            \centering
            \includegraphics[width=\textwidth]{Figure_1.png}
            \caption{L1与L2正则化的几何解释}
        \end{figure}

        如图,以 $d=2$ 维特征 $w_1, w_2$ 为例。
        \begin{itemize}
            \item $L(\mathbf{w})$ 等高线(椭圆): 图中的椭圆形等高线代表原始损失函数 $L(\mathbf{w})$。越靠近中心的椭圆,损失越小。中心点 $\hat{\mathbf{w}}_{OLS}$ 是没有正则化时的最优解(即过拟合的解)。
            \item L1 约束区域(菱形): L1 范数的约束 $P_{L1}(\mathbf{w}) \le C$(即 $|w_1| + |w_2| \le C$)在二维平面上是一个菱形(或旋转的正方形)。
            \item L2 约束区域(圆形): L2 范数的约束 $P_{L2}(\mathbf{w}) \le C$(即 $w_1^2 + w_2^2 \le C$)在二维平面上是一个圆形。
        \end{itemize}
         L1 正则化倾向于使最优解 $\mathbf{w}$ 落在坐标轴上,导致很多特征的权重 $w_j$ 精确地等于 0。 L2 正则化会使所有权重 $w_j$ 趋近于 0,但通常不会等于 0。它倾向于将权重“平均”分配给相关的特征。

         \item 相同点
         \begin{itemize}
            \item 两者都建立在“信号是可稀疏表示的”这一核心假设之上。
            \item 在数学上,两者都经常依赖 L0 范数(非零元素个数)来度量稀疏性,并由于 L0 的 NP-hard 性质,转而使用其凸松弛 L1 范数 ( $\|\alpha\|_1$ ) 进行优化求解。
         \end{itemize}
            不同点\begin{itemize}
                \item 字典学习的目标是学习一个字典 $D$,能让数据 $X$ 得到最稀疏的表示.
                \item 压缩感知的目标是重建一个信号 $x$。稀疏性在这里是一种先验知识。它是信号 $x$ 固有的一个属性,是使得从欠定方程 $y = \Phi x$ 中解出 $x$ 成为可能的前提。
            \end{itemize}
         
    \end{enumerate}
\end{solution*}

\section{四. (20 points) 半监督 SVM  }

考虑一个二分类问题,其中我们有一组标记数据集 $\mathcal{D}_{L}=\{(x_{1},y_{1}),(x_{2},y_{2}),\ldots,(x_{l},y_{l})\}$ 和一组未标记数据集 $\mathcal{D}_{U}=\{x_{l+1},x_{l+2},\ldots,x_{m}\}$,其中 $y_{i}\in\{-1,+1\}$ 为标记数据的类别。假设数据集 $\{x_{1},x_{2},\ldots,x_{m}\}$ 来自于某个未知的分布,并且 TSVM 的目标是通过最小化带有约束的目标函数来学习分类器。在 TSVM 中,我们同时优化标记数据的分类误差和无标记数据的决策边界,使得无标记数据尽可能被正确分类。  
\begin{enumerate}
    \item 给出 TSVM 的目标函数并且给出 TSVM 的约束条件。(5 分)  

\item 使用拉格朗日乘子法,推导 TSVM 的对偶问题。首先写出拉格朗日函数,并根据拉格朗日乘子法推导出对偶问题的目标函数。注意:在推导过程中,假设未标记样本的标签 $y_u$ 暂时固定为常量。(9 分)  

提示:  
\begin{itemize}
    \item 使用拉格朗日乘子法将约束条件引入目标函数。
\item 对每个约束条件引入拉格朗日乘子,并对拉格朗日函数进行偏导数计算,得到对偶问题。
\item 最终的对偶问题将是一个与原始空间(原始变量)相关的优化问题。 
\end{itemize}
 

\item 解释为什么当未标记样本的标签也作为优化变量时, TSVM 的优化开销极大? 常见的优化策略是什么? (参考讲义第 13 章第 20 页 pseudocode 回答问题) (6 分)
\end{enumerate}
  
\begin{solution*}
    \begin{enumerate}
        \item 原始目标函数是:$$\min_{\mathbf{w}, b, \boldsymbol{\xi}} \quad \frac{1}{2}\|\mathbf{w}\|^2 + C_L \sum_{i=1}^l \xi_i + C_U \sum_{u=l+1}^m \xi_u$$
        
        约束条件 (Constraints) 是:$$y_i(\mathbf{w}^\top \mathbf{x}_i + b) \ge 1 - \xi_i, \quad i=1, \ldots, l$$$$y_u(\mathbf{w}^\top \mathbf{x}_u + b) \ge 1 - \xi_u, \quad u=l+1, \ldots, m$$$$\xi_i \ge 0, \quad i=1, \ldots, m$$
        \item 拉格朗日函数为:$$\mathcal{L}(\mathbf{w}, b, \boldsymbol{\xi}, \boldsymbol{\alpha}, \boldsymbol{\beta})= \frac{1}{2}\|\mathbf{w}\|^2 + \sum_{i=1}^m C_i \xi_i - \sum_{i=1}^m \alpha_i [y_i(\mathbf{w}^\top \mathbf{x}_i + b) - 1 + \xi_i] - \sum_{i=1}^m \beta_i \xi_i$$
        
        分别对 $\mathbf{w}, b, \xi_i$ 求偏导:
        $$\frac{\partial \mathcal{L}}{\partial \mathbf{w}} = \mathbf{w} - \sum_{i=1}^m \alpha_i y_i \mathbf{x}_i = 0 \implies \mathbf{w} = \sum_{i=1}^m \alpha_i y_i \mathbf{x}_i$$
        $$\frac{\partial \mathcal{L}}{\partial b} = - \sum_{i=1}^m \alpha_i y_i = 0 \implies \sum_{i=1}^m \alpha_i y_i = 0$$
        $$\frac{\partial \mathcal{L}}{\partial \xi_i} = C_i - \alpha_i - \beta_i = 0 \implies C_i = \alpha_i + \beta_i$$

        代入拉格朗日函数得\[\frac{1}{2}\|\w\|^2-\sum_{i=1}^{m}\alpha_iy_i\w^\top\x_i=\frac{1}{2}\|\w\|^2-\w^\top\left(\sum_{i=1}^{m}\alpha_iy_i\x_i\right)=\frac{1}{2}\|\w\|^2-\w^\top\w=-\frac{1}{2}\|\w\|^2\]\[\L=\sum_{i=1}^{m}\alpha_i-\frac{1}{2}\|\w\|^2=\sum_{i=1}^{m}\alpha_i-\frac{1}{2}\sum_{i=1}^{m}\sum_{j=1}^{m}\alpha_i\alpha_jy_iy_j(\x_i^\top\x_j)\]

        对偶问题为:$$\max_{\boldsymbol{\alpha}} \quad \sum_{i=1}^m \alpha_i - \frac{1}{2} \sum_{i=1}^m \sum_{j=1}^m \alpha_i \alpha_j y_i y_j  \mathbf{x}_i^\top \mathbf{x}_j$$
        $$\sum_{i=1}^m \alpha_i y_i = 0$$
        $$0 \le \alpha_i \le C_L, \quad i=1, \ldots, l$$$$0 \le \alpha_i \le C_U, \quad i=l+1, \ldots, m$$

        \item 开销大的原因:优化目标函数在 $\mathbf{w}, b, \boldsymbol{\xi}$ 上是凸的(对于固定的 $\hat{\mathbf{y}}_U$),但在 $\hat{\mathbf{y}}_U$ 上是非凸的、离散的。
要找到全局最优解,理论上需要尝试所有 $2^k$ 种标签组合,为每种组合求解一个标准的凸 SVM 问题,然后比较 $2^k$ 个结果的原始目标函数值。
这种 $O(2^k)$ 的计算复杂度是指数级的,使得该问题成为 NP-Hard 问题。

优化策略:如13章第20页的伪代码所示。先仅使用标记数据训练一个初始 SVM 分类器,然后使用该分类器对未标记数据进行预测,得到初始的伪标签。初始化折中参数$C_u<<C_l$.基于标签和伪标签来优化SVM,得到$(\w,b),\xi$.

对于相反的伪标签$\hat{y}_i,\hat{y}_j$,如果$\xi_i>0$(i进入间隔或者错分)且$\xi_j>0$(j进入间隔或者错分)且$\xi_i+\xi_j>2$(至少有一个被严重错分),则交换它们的伪标签,并重新求解SVM问题,得到新的$(\w,b),\xi$.重复上述过程,直到没有伪标签交换为止。

然后将$C_u$增加一倍,重复上述过程,直到$C_u\ge C_l$为止。
    \end{enumerate}
\end{solution*}

\section{五. (20 points) EM 算法及其应用  }
\begin{enumerate}

\item 广义 EM 算法 (5 分)  

设 x 为观测数据,z 为潜在变量,$\theta$为参数。证明在广义 EM 算法中,不完全数据的对数似然函数 $l(\theta; x)$ 是非递减的。  

\item 条件混合模型的构建与 EM 算法应用(15 分)  

最简单的构造条件混合模型的方法是直接将非条件混合模型(例如,混合高斯模型)中的密度替换 为条件分布。在这个问题中,我们考虑一种简单的线性回归模型的混合模型。假设我们希望使用包 含 c 个线性回归模型的混合模型来拟合数据集 $\left\{\left(x_{i}, y_{i}\right): x_{i} \in \mathbb{R}^{p}, y_{i} \in \mathbb{R}, i=1, \ldots, n\right\}$ 。每个回归 模型有自己的回归系数 $w_{k} \in \mathbb{R}^{p}$ 和混合系数 $\pi_{k}$ 。为了简化,假设所有回归模型共享方差 $\sigma^{2}$ 。设 $\Theta=\left\{W, \pi, \sigma^{2}\right\}$ 为参数集,其中 $W=\left\{w_{k}\right\}, \pi=\left\{\pi_{k}\right\}$ 。类似于高斯混合模型(GMM),我们可以 引入一个 c 维的二元潜在变量 $z_{i}$ ,其中只有一个非零元素。响应变量 $y_{i}$ 的生成过程可以写成:  
\begin{itemize}

\item 从多项分布 $Mult(\pi)$ 中抽取指示变量 $z_i$;  

\item 从高斯分布 $N(w_{z_i}^\top x_i, \sigma^2)$ 中抽取响应变量 $y_i$,其中 $w_{z_i}$ 表示由 $z_i$ 选中的回归系数。  
\end{itemize}
接下来,我们将推导出一个EM 算法来获得这个模型的最大似然估计(MLE)。具体来说  
\begin{enumerate}[(1)]
\item  推导出不完全数据对数似然函数的下界;

\item  推导 E 步, 使用 $r_{ik}$ 作为 “责任” 变量; $r_{ik} = p(z_{ik} = 1 \mid x_i, y_i, \Theta)$

\item  推导 M 步, 给出参数 $\Theta = \{W, \pi, \sigma^2\}$ 的更新规则。  
\end{enumerate}
提示:请参考讲义第14章第46页GMM EM推导的形式与步骤。类比GMM中的latent variable $z_i$ 和soft assignment (responsibility),我们将密度$\mathcal{N}(x|\mu_k,\Sigma_k)$替换为条件分布$\mathcal{N}(y_i\mid w_k^\top x_i,\sigma^2)$。
\end{enumerate}

\begin{solution*}
    \begin{enumerate}
        \item 不完全数据的对数似然函数 $l(\theta; x) = \log p(x | \theta)$。引入潜在变量 $z$ 后,我们有 $p(x | \theta) = \frac{p(x, z | \theta)}{p(z | x, \theta)}$。因此,$l(\theta; x) = \log p(x, z | \theta) - \log p(z | x, \theta)$。在给定当前参数 $\theta^{(t)}$ 和观测数据 $x$ 的条件下,我们对上式两边取关于 $z$ 的期望 $E_{z | x, \theta^{(t)}}[\cdot]$:
        \begin{equation}
            \label{eq:1}
            E_{z | x, \theta^{(t)}} [l(\theta; x)] = E_{z | x, \theta^{(t)}} [\log p(x, z | \theta)] - E_{z | x, \theta^{(t)}} [\log p(z | x, \theta)]
        \end{equation}
        
        
        由于 $l(\theta; x)$ 是关于 $z$ 的常数,其期望就是它自身。
        \[l(\theta; x)=E_{z | x, \theta^{(t)}} [l(\theta; x)]\]
        定义:$$Q(\theta, \theta^{(t)}) = E_{z | x, \theta^{(t)}} [\log p(x, z | \theta)]$$
        $$H(\theta, \theta^{(t)}) = E_{z | x, \theta^{(t)}} [\log p(z | x, \theta)]$$
        
        式\ref{eq:1}可化为:$$l(\theta; x) = Q(\theta, \theta^{(t)}) - H(\theta, \theta^{(t)})$$

        要证明$l$非递减,即证明$$l(\theta^{(t+1)}; x) - l(\theta^{(t)}; x) = [Q(\theta^{(t+1)}, \theta^{(t)}) - Q(\theta^{(t)}, \theta^{(t)})] - [H(\theta^{(t+1)}, \theta^{(t)}) - H(\theta^{(t)}, \theta^{(t)})]>0$$

        考虑Q部分,M步对Q进行最大化,所以Q必然是非递减的,$$Q(\theta^{(t+1)}, \theta^{(t)}) - Q(\theta^{(t)}, \theta^{(t)})\ge0$$

        考虑H部分:$$H(\theta^{(t+1)}, \theta^{(t)}) - H(\theta^{(t)}, \theta^{(t)}) = E_{z | x, \theta^{(t)}} [\log p(z | x, \theta^{(t+1)})] - E_{z | x, \theta^{(t)}} [\log p(z | x, \theta^{(t)})]$$

        根据吉布斯不等式,对于任意两个概率分布 $P(z)$ 和 $q(z)$, $-KL(P \parallel q) \le 0$,即 $E_P[\log q(z)] \le E_P[\log P(z)]$。令 $P(z) = p(z | x, \theta^{(t)})$ 且 $q(z) = p(z | x, \theta^{(t+1)})$,我们得到:$$E_{z | x, \theta^{(t)}} \left[ \log p(z | x, \theta^{(t+1)}) \right] \le E_{z | x, \theta^{(t)}} \left[ \log p(z | x, \theta^{(t)}) \right]$$
        因此,$$[H(\theta^{(t+1)}, \theta^{(t)}) - H(\theta^{(t)}, \theta^{(t)})] \le 0$$

        所以\[l(\theta^{(t+1)}; x) - l(\theta^{(t)}; x) \ge 0\]

        \item \begin{enumerate}[(1)]
            \item 对于单个样本来说,似然函数$$p(y_i | x_i, \Theta) = \sum_{k=1}^c p(z_{ik}=1, y_i | x_i, \Theta)= \sum_{k=1}^c p(z_{ik}=1 | \Theta) p(y_i | x_i, z_{ik}=1, \Theta)$$根据模型定义, $p(z_{ik}=1 | \Theta) = \pi_k$ 且 $p(y_i | x_i, z_{ik}=1, \Theta) = \mathcal{N}(y_i | w_k^\top x_i, \sigma^2)$。$$p(y_i | x_i, \Theta) = \sum_{k=1}^c \pi_k \mathcal{N}(y_i | w_k^\top x_i, \sigma^2)$$所以不完全数据对数似然函数 $l(\Theta)$ 为:$$l(\Theta) = \sum_{i=1}^n \log p(y_i | x_i, \Theta) = \sum_{i=1}^n \log \left( \sum_{k=1}^c \pi_k \mathcal{N}(y_i | w_k^\top x_i, \sigma^2) \right)$$
            
            完全数据对数似然
$l_c(\Theta) = \log p(Y, Z | X, \Theta) = \log \prod_{i=1}^n p(y_i, z_i | x_i, \Theta)$

$$p(y_i, z_i | x_i, \Theta) = \prod_{k=1}^c \left[ p(z_{ik}=1 | \Theta) p(y_i | x_i, z_{ik}=1, \Theta) \right]^{z_{ik}}= \prod_{k=1}^c \left[ \pi_k \mathcal{N}(y_i | w_k^\top x_i, \sigma^2) \right]^{z_{ik}}$$
$$l_c(\Theta) = \sum_{i=1}^n \log \left( \prod_{k=1}^c \left[ \pi_k \mathcal{N}(y_i | w_k^\top x_i, \sigma^2) \right]^{z_{ik}} \right)= \sum_{i=1}^n \sum_{k=1}^c z_{ik} \left( \log \pi_k + \log \mathcal{N}(y_i | w_k^\top x_i, \sigma^2) \right)$$
Q函数是 $l_c(\Theta)$ 在给定 $X, Y$ 和当前参数 $\Theta^{(t)}$ 下对 $Z$ 的期望:$$Q(\Theta, \Theta^{(t)}) = E_{Z | X, Y, \Theta^{(t)}} [l_c(\Theta)]$$
因为$$E[z_{ik}] = p(z_{ik}=1 | x_i, y_i, \Theta^{(t)}) \triangleq r_{ik}$$
所以不完全数据对数似然函数的下界 (Q-函数) 为:$$Q(\Theta, \Theta^{(t)}) = \sum_{i=1}^n \sum_{k=1}^c r_{ik} \left( \log \pi_k + \log \mathcal{N}(y_i | w_k^\top x_i, \sigma^2) \right)$$

\item \begin{align*}
    r_{ik} =& p(z_{ik}=1 | x_i, y_i, \Theta^{(t)})\\
=& \frac{p(z_{ik}=1, y_i | x_i, \Theta^{(t)})}{p(y_i | x_i, \Theta^{(t)})}\\
=& \frac{p(z_{ik}=1 | \Theta^{(t)}) p(y_i | x_i, z_{ik}=1, \Theta^{(t)})}{\sum_{j=1}^c p(z_{ij}=1 | \Theta^{(t)}) p(y_i | x_i, z_{ij}=1, \Theta^{(t)})}\\
=& \frac{\pi_k^{(t)} \mathcal{N}(y_i | (w_k^{(t)})^\top x_i, (\sigma^2)^{(t)})}{\sum_{j=1}^c \pi_j^{(t)} \mathcal{N}(y_i | (w_j^{(t)})^\top x_i, (\sigma^2)^{(t)})}
\end{align*}

\item \begin{align*}
    Q(\Theta, \Theta^{(t)}) &= \sum_{i=1}^n \sum_{k=1}^c r_{ik} \log \pi_k + \sum_{i=1}^n \sum_{k=1}^c r_{ik} \log \mathcal{N}(y_i | w_k^\top x_i, \sigma^2)\\
    &= \sum_{i=1}^n \sum_{k=1}^c r_{ik} \log \pi_k + \sum_{i=1}^n \sum_{k=1}^c r_{ik} \left[ -\frac{1}{2}\log(2\pi\sigma^2) - \frac{(y_i - w_k^\top x_i)^2}{2\sigma^2} \right]\\
    &= \sum_{i=1}^n \sum_{k=1}^c r_{ik} \log \pi_k - \frac{n}{2}\log(2\pi) - \frac{n}{2}\log(\sigma^2) - \frac{1}{2\sigma^2}\sum_{i=1}^n \sum_{k=1}^c r_{ik}(y_i - w_k^\top x_i)^2
\end{align*}

更新$\pi_k$:
$$\mathcal{L}(\pi, \lambda) = \sum_{i=1}^n \sum_{k=1}^c r_{ik} \log \pi_k + \lambda (1 - \sum_{k=1}^c \pi_k)$$
对$\pi_k$求导并令其为0:$$\frac{\partial \mathcal{L}}{\partial \pi_k} = \sum_{i=1}^n \frac{r_{ik}}{\pi_k} - \lambda = 0 \implies \pi_k = \frac{\sum_{i=1}^n r_{ik}}{\lambda}$$
因为\[\sum_{k=1}^{n}\pi_k = \frac{\sum_{k=1}^{n}\sum_{i=1}^n r_{ik}}{\lambda}=\frac{n}{\lambda}=1\]
所以$$\boxed{\pi_k^{(t+1)} =\frac{1}{n} \sum_{i=1}^n r_{ik}}$$

更新 $w_k$:

最大化 $Q$ 中关于 $w_k$ 的部分。这等价于最小化 $Q$ 中关于 $w_k$ 的负向部分:$$\min_{w_k} \quad J(w_k) = \frac{1}{2\sigma^2}\sum_{i=1}^n \sum_{k=1}^c r_{ik}(y_i - w_k^\top x_i)^2$$由于我们是分别为每个 $k$ 更新 $w_k$,我们只需最小化:$$\min_{w_k} \quad J(w_k) = \sum_{i=1}^n r_{ik}(y_i - w_k^\top x_i)^2$$
对其求梯度 $\nabla_{w_k} J(w_k)$ 并置零:$$\nabla_{w_k} J(w_k) = \sum_{i=1}^n r_{ik} \cdot 2 (y_i - w_k^\top x_i) \cdot (-x_i) = 0$$
$$\sum_{i=1}^n r_{ik} (w_k^\top x_i) x_i = \sum_{i=1}^n r_{ik} y_i x_i$$
$$\boxed{\left( \sum_{i=1}^n r_{ik} x_i x_i^\top \right) w_k^{(t+1)} = \sum_{i=1}^n r_{ik} x_i y_i}$$

更新 $\sigma^2$:

最大化 $Q$ 中关于 $\sigma^2$ 的部分。使用 M 步更新后的 $w_k^{(t+1)}$:$$Q_{\sigma^2} = - \frac{n}{2}\log(\sigma^2) - \frac{1}{2\sigma^2}\sum_{i=1}^n \sum_{k=1}^c r_{ik}(y_i - (w_k^{(t+1)})^\top x_i)^2$$
对 $\sigma^2$ 求导并置零:$$\frac{\partial Q_{\sigma^2}}{\partial (\sigma^2)} = - \frac{n}{2\sigma^2} - (-1) \frac{1}{2(\sigma^2)^2} \sum_{i=1}^n \sum_{k=1}^c r_{ik}(y_i - (w_k^{(t+1)})^\top x_i)^2 = 0$$

$$\boxed{(\sigma^2)^{(t+1)} = \frac{1}{n} \sum_{i=1}^n \sum_{k=1}^c r_{ik}(y_i - (w_k^{(t+1)})^\top x_i)^2}$$
        \end{enumerate}
    \end{enumerate}
\end{solution*}
\end{document}