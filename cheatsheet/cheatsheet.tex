\documentclass[10pt,a4paper,landscape]{article}
\usepackage{multicol}
\usepackage{calc}
\usepackage{ifthen}
\usepackage[landscape]{geometry}
\usepackage{amsmath,amsthm,amsfonts,amssymb}
\usepackage{color,graphicx,overpic}
\usepackage{hyperref}
\usepackage{enumitem}
\usepackage{bm}
\usepackage{xeCJK} % 处理中文
\setlength{\parindent}{0em}
% 页面设置:极小边距,最大化利用空间
\geometry{top=0.5cm,left=0.5cm,right=0.5cm,bottom=0.5cm}

% 字体设置
% 设置中文字体:主字体为宋体,粗体自动调用黑体,并开启伪粗体支持
\setCJKmainfont[BoldFont=SimHei, AutoFakeBold=true]{SimSun}
\setmainfont{Times New Roman}

% 自定义环境
\pagestyle{empty}
\makeatletter
\renewcommand{\section}{\@startsection{section}{1}{0mm}%
                                {-1ex plus -.5ex minus -.2ex}%
                                {0.5ex plus .2ex}%x
                                {\normalfont\large\bfseries\color{red}}}
\renewcommand{\subsection}{\@startsection{subsection}{2}{0mm}%
                                {-1explus -.5ex minus -.2ex}%
                                {0.5ex plus .2ex}%
                                {\normalfont\normalsize\bfseries\color{blue}}}
\renewcommand{\subsubsection}{\@startsection{subsubsection}{3}{0mm}%
                                {-1ex plus -.5ex minus -.2ex}%
                                {1ex plus .2ex}%
                                {\normalfont\small\bfseries}}
\makeatother

% 自定义高亮命令:加粗且黄色底
\newcommand{\hl}[1]{\colorbox{yellow}{\textbf{#1}}}
% 公式高亮:绿色背景 (替代 $...$)
\newcommand{\mhl}[1]{\begingroup\setlength{\fboxsep}{0pt}\colorbox{green}{$#1$}\endgroup}
% 公式强调:红色加粗 (可在公式内使用)
\newcommand{\mred}[1]{{\color{red}\bm{#1}}}

% 紧凑列表
\setlist[itemize]{leftmargin=*,noitemsep,topsep=0pt}
\setlist[enumerate]{leftmargin=*,noitemsep,topsep=0pt}

% 重新定义公式间距
\setlength{\abovedisplayskip}{2pt}
\setlength{\belowdisplayskip}{2pt}

\begin{document}
% \footnotesize % 全局字体缩小
\linespread{0.99}\selectfont % 进一步压缩行距

\begin{multicols*}{3} % 三栏排版

% ==========================================
% 第一部分:基础与评估
% ==========================================
\section{基础与评估}
\subsection{偏差-方差分解}
泛化误差可分解为偏差、方差与噪声之和:
$ E(f;D) = \text{bias}^2(\bm{x}) + \text{var}(\bm{x}) + \varepsilon^2 $。
 \textbf{偏差} $\text{bias}^2(\bm{x}) = (\bar{f}(\bm{x}) - y)^2$:度量学习算法的期望预测与真实结果的偏离程度(拟合能力)。\textbf{方差} $\text{var}(\bm{x}) = \mathbb{E}_D[(f(\bm{x};D) - \bar{f}(\bm{x}))^2]$:度量同样大小的训练集变动导致的性能变化(稳定性)。\textbf{噪声} $\varepsilon^2$:数据本身的难度。

\subsection{高斯分布}
$\mathcal{N}(\bm{x}|\bm{\mu}_i, \bm{\Sigma}_i)$,其中 $\mathcal{N}(\bm{x}|\bm{\mu}, \bm{\Sigma}) = \frac{1}{(2\pi)^{d/2}|\bm{\Sigma}|^{1/2}} \exp(-\frac{1}{2}(\bm{x}-\bm{\mu})^T\bm{\Sigma}^{-1}(\bm{x}-\bm{\mu}))$。

\subsection{性能度量}
\textbf{查准率(Precision)}: $P = \frac{TP}{TP+FP}$
\textbf{查全率(Recall)}: $R = \frac{TP}{TP+FN}$
\textbf{F1-Score}: $\frac{2 \times P \times R}{P + R}$
\textbf{ROC与AUC}: ROC曲线坐标为(FPR, TPR),AUC为曲线下覆盖面积,衡量排序质量。

% ==========================================
% 第二部分:线性模型与SVM
% ==========================================
\section{线性模型与SVM}
\subsection{逻辑回归 (Logistic Regression)}
使用对数几率函数(Sigmoid)逼近后验概率:
$ y = \frac{1}{1+e^{-(\bm{w}^T\bm{x}+b)}} $
对数几率 $\ln \frac{y}{1-y} = \bm{w}^T\bm{x}+b$。
优化目标(最大化对数似然):
$ \min_{\bm{w},b} \sum_{i=1}^m \ln(1+e^{-y_i(\bm{w}^T\bm{x}_i+b)}) $
(假设 $y_i \in \{-1, +1\}$ 或对应调整公式)。

\subsection{支持向量机 (SVM)}
\textbf{基本型}:最大化间隔 $\gamma = \frac{2}{||\bm{w}||}$。
$ \min_{\bm{w},b} \frac{1}{2}||\bm{w}||^2, \quad s.t.\ y_i(\bm{w}^T\bm{x}_i+b) \ge 1 $
\textbf{对偶问题}:引入拉格朗日乘子 $\alpha_i \ge 0$:
$ \max_{\bm{\alpha}} \sum_{i=1}^m \alpha_i - \frac{1}{2}\sum_{i=1}^m\sum_{j=1}^m \alpha_i\alpha_j y_i y_j \bm{x}_i^T \bm{x}_j $
$ s.t.\ \sum_{i=1}^m \alpha_i y_i = 0, \quad \alpha_i \ge 0 $
解得 $\bm{w} = \sum_{i=1}^m \alpha_i y_i \bm{x}_i$。支持向量满足 $\alpha_i > 0$。
\textbf{软间隔}:允许部分样本出错,引入松弛变量 $\xi_i$ 和惩罚参数 $C$:
$ \min_{\bm{w},b} \frac{1}{2}||\bm{w}||^2 + C\sum_{i=1}^m \xi_i $
\textbf{核技巧}:$\kappa(\bm{x}_i, \bm{x}_j) = \phi(\bm{x}_i)^T\phi(\bm{x}_j)$,解决非线性可分。常用高斯核 $\kappa(\bm{x},\bm{y}) = \exp(-\frac{||\bm{x}-\bm{y}||^2}{2\sigma^2})$。

% ==========================================
% 第三部分:集成学习
% ==========================================
\section{集成学习}
\hl{误差-分歧分解}:$E = \bar{E} - \bar{A}$ (集成误差=个体平均误差-个体平均分歧)。
\textbf{AdaBoost}
最小化指数损失 $L(y, f(x)) = e^{-y f(x)}$。
\textbf{更新权重}:若 $h_t$ 错误率 $\epsilon_t$,则权重 $\alpha_t = \frac{1}{2}\ln \frac{1-\epsilon_t}{\epsilon_t}$。
样本分布更新:错分样本权重增加 $D_{t+1}(\bm{x}) \propto D_t(\bm{x}) e^{-\alpha_t y_i h_t(\bm{x}_i)}$。

\hl{Bagging \& Random Forest}
Bagging: Bootstrap采样训练基学习器,投票/平均。
Random Forest: Bagging + 属性随机选择(基尼指数选择划分时只考虑随机子集)。

% ==========================================
% 第四部分:聚类
% ==========================================
\section{聚类}
\subsection{距离度量}
四个性质:非负性、同一性、对称性、直递性

无序属性:令$m_{u,a}$表示属性 u 上取值为 a 的样本数,$m_{u,a,i}$表示在第 i 个样本簇中在属性 u 上取值为 a 的样本数,k 为样本簇数,则属性 u 上两个离散值 a 与 b 之间的 VDM 距离为
\mhl{VDM_p(a,b)=\sum_{i=1}^{k}\bigg|\frac{m_{u,a,i}}{m_{u,a}} - \frac{m_{u,b,i}}{m_{u,b}}\bigg|^p}

\subsection{分类}
\begin{itemize}
    \item \textbf{原型聚类}:k-means,学习向量量化,高斯混合聚类\begin{itemize}
        \item \hl{k-means}:1. 初始化 $k$ 个中心 $\bm{\mu}_j$。
        2. E步:分配样本到最近中心 $C_j = \{ \bm{x}_i | ||\bm{x}_i - \bm{\mu}_j|| \le ||\bm{x}_i - \bm{\mu}_{j'}|| \}$.
        3. M步:更新中心 $\bm{\mu}_j = \frac{1}{|C_j|} \sum_{\bm{x} \in C_j} \bm{x}$。
        4. 重复直到收敛。
        \item \hl{学习向量量化 (LVQ)}:约等于k-means但是数据带标记。计算每个样本到各中心p的距离,如果x和p标签相同,$p'=p+\eta (x-p)$,否则$p'=p-\eta (x-p)$。
        \item \hl{高斯混合模型 (GMM)}:假设 $P(\bm{x}) = \sum_{i=1}^k \alpha_i p(\bm{x}|\bm{\mu}_i, \bm{\Sigma}_i)$。
使用 \textbf{EM算法} 求解:
1. \textbf{最大化似然函数}$LL=\sum_{j=1}^{m}\ln(\sum_{i=1}^{k}\alpha_ip(x_j|\mu_i,\Sigma_i))$ 
2.\textbf{E步}:计算后验概率 $\gamma_{ji} = P(z_j=i|\bm{x}_j) = \frac{\alpha_i p(\bm{x}_j|\bm{\mu}_i,\bm{\Sigma}_i)}{\sum_l \alpha_l p(\bm{x}_j|\bm{\mu}_l,\bm{\Sigma}_l)}$。
3. \textbf{M步}:更新参数 $\bm{\mu}_i, \bm{\Sigma}_i, \alpha_i$。$\bm{\mu}_i^{new} = \frac{1}{N_i} \sum_{j=1}^m \gamma_{ji} \bm{x}_j$。$\bm{\Sigma}_i^{new} = \frac{1}{N_i} \sum_{j=1}^m \gamma_{ji} (\bm{x}_j - \bm{\mu}_i^{new})(\bm{x}_j - \bm{\mu}_i^{new})^T$。$\alpha_i^{new} = \frac{N_i}{m}$

    \end{itemize}
    \item \textbf{密度聚类}:DBSCAN,OPTICS,DENCLUE
    \begin{itemize}
        \item \hl{DBSCAN}:\textbf{核心对象}:邻域内至少包含 MinPts 个样本的点。
\textbf{密度直达}:位于核心对象邻域内的点。
\textbf{密度可达}:存在一条由密度直达连接的路径。
\textbf{密度相连}:存在共同密度可达的点。
    \end{itemize}
    \item \textbf{层次聚类}:AGNES,DIANA
    \begin{itemize}
        \item \hl{AGNES}:
每个样本作为一个簇,合并两个最近的簇直到大一统。
    \end{itemize}
\end{itemize}

% ==========================================
% 第五部分:降维与特征选择
% ==========================================
\section{降维}
\subsection{PCA (主成分分析)}
目标:最近重构性 或 最大可分性。
解:协方差矩阵 $\bm{X}\bm{X}^T$ 的前 $d'$ 个最大特征值对应的特征向量。
$ \bm{X}\bm{X}^T \bm{w}_i = \lambda_i \bm{w}_i $
重构:$\hat{\bm{x}} = \sum_{i=1}^{d'} z_i \bm{w}_i$。

\textbf{步骤}:1. 数据中心化$\hat{X}=X-\bar{X}$. 2. 计算协方差矩阵 $\Sigma=\hat{X}^T\hat{X}$。3. 求解特征值与特征向量. $\det(\Sigma -\lambda I)=0$解得特征值,依次代入特征值计算$(\Sigma-\lambda I)v=0$得到特征向量。4. 选择前 $d'$ 个特征向量构成投影矩阵 $\bm{W} = [\bm{v}_1, \ldots, \bm{v}_{d'}]$。5. 投影降维:$\bm{Z} = \hat{\bm{X}}\bm{W}$。
\subsection{流形学习}
\textbf{ISOMAP}:MDS + 测地线距离(最短路径算法)。
\textbf{LLE}:保持局部线性关系(重构权重)。
\begin{enumerate}
    \item \textbf{找近邻}:对 $\bm{x}_i$ 找 $k$ 近邻集合 $Q_i$。
    \item \textbf{算权重}:最小化重构误差求权重 $w_{ij}$(保持 $\bm{x}_i$ 由邻居线性表示)。
    $ \min_{\bm{w}} \sum_{i} ||\bm{x}_i - \sum_{j \in Q_i} w_{ij} \bm{x}_j||^2, \quad s.t. \sum_{j \in Q_i} w_{ij} = 1 $
    \item \textbf{求坐标}:固定 $w_{ij}$,求低维坐标 $\bm{z}_i$。
    $ \min_{\bm{Z}} \sum_{i} ||\bm{z}_i - \sum_{j \in Q_i} w_{ij} \bm{z}_j||^2 = \text{tr}(\bm{Z} \bm{M} \bm{Z}^T) $
    其中 $\bm{M} = (\bm{I}-\bm{W})^T (\bm{I}-\bm{W})$。
    \item \textbf{解}:$\bm{M}$ 的最小 $d'$ 个非零特征值对应的特征向量。
\end{enumerate}
\subsection{距离度量学习}
\textbf{核心}:学习马氏距离矩阵 $\bm{M}$,使度量适应任务。
$ \text{dist}_{\text{mah}}^2(\bm{x}_i, \bm{x}_j) = (\bm{x}_i - \bm{x}_j)^T \bm{M} (\bm{x}_i - \bm{x}_j) = ||\bm{x}_i - \bm{x}_j||_{\bm{M}}^2 $
其中 $\bm{M} \succeq 0$(半正定对称矩阵)。

\textbf{基于约束的方法}:
给定必连集合 $\mathcal{M}$ (同类) 和勿连集合 $\mathcal{C}$ (异类),求解凸优化:
$ \min_{\bm{M}} \sum_{(\bm{x}_i, \bm{x}_j) \in \mathcal{M}} ||\bm{x}_i - \bm{x}_j||_{\bm{M}}^2 ~~
 s.t. \sum_{(\bm{x}_i, \bm{x}_k) \in \mathcal{C}} ||\bm{x}_i - \bm{x}_k||_{\bm{M}}^2 \ge 1, \quad \bm{M} \succeq 0 $

\textbf{代表性算法}
\begin{itemize}
    \item \textbf{NCA (近邻成分分析)}:优化随机近邻分类器的留一法(LOO)正确率。概率 $p_{ij} \propto \exp(-||\bm{x}_i - \bm{x}_j||_{\bm{M}}^2)$。
    \item \textbf{LMNN (最大间隔近邻)}:
    \begin{itemize}
        \item \textbf{拉近}:最小化同类 $k$ 近邻的距离。
        \item \textbf{推开}:保证异类样本与同类近邻之间有间隔 (Margin)。
    \end{itemize}
\end{itemize}
\section{特征选择与稀疏学习}
\subsection{特征选择}
\textbf{信息增益公式}:
    \mhl{ Gain(A) = Ent(D) - \sum_{v=1}^V \frac{|D^v|}{|D|} Ent(D^v) } 
    其中 \mhl{ Ent(D) = -\sum_{k=1}^K \frac{|D_k|}{|D|} \log_2 \frac{|D_k|}{|D|} }。
\begin{itemize}
    \item \textbf{过滤式 (Filter)-Relief}:先用特征选择过程过滤原始数据,再进行模型训练;特征选择与后续学习器无关
    \item \textbf{包裹式 (Wrapper)-LVW}:直接针对给定学习器进行优化,“量身定做”特征子集。性能通常更好,但计算开销大
    \item \textbf{嵌入式 (Embedded)-LASSO\&PGD}:将特征选择过程与学习器训练过程融为一体,通常通过 \textbf{正则化}(岭回归($+\lambda\frac{1}{2}\|w\|^2_2$),LASSO($+\lambda\|w\|_1$)) 实现
\end{itemize}
\subsection{稀疏学习}
\begin{itemize}
    \item \textbf{字典学习}:将稠密数据转化为稀疏表示。学习字典$B$和稀疏系数$A$.优化目标:$ \min_{B, \alpha_i} \sum_{i=1}^m \|x_i - B \alpha_i\|_2^2 + \lambda \|\alpha_i\|_1 $。交替优化B和$\alpha_i$。
    \item \textbf{压缩感知}:利用部分采样恢复全信号。$y = \Phi x$。条件:信号稀疏,采样矩阵满足限定等距性。求解$\min_s \|s\|_1 \quad \text{s.t.} \quad y = \Phi \Psi s$,从y中恢复稀疏信号s,进而恢复x。
    \item \textbf{矩阵补全}:解决推荐系统等场景下的数据缺失问题。优化:$\min_{X} \|X\|_* \quad \text{s.t.} \quad X_{ij} = A_{ij}, \quad (i,j) \in \Omega$。其中$\|X\|_* = \sum \sigma_i$。
\end{itemize}

% ==========================================
% 第六部分:半监督学习
% ==========================================
\section{半监督学习}
假设:聚类假设(同簇同类)、流形假设(邻近同类)。
\subsection{生成式方法}
假设 $P(\bm{x},y) = \sum_{k} \alpha_k p(\bm{x}|\bm{\theta}_k)$,将未标记数据视为隐变量,最大化对数似然函数:
$
    \ln p(D_l \cup D_u) = \sum_{(x_j, y_j) \in D_l} \ln (\sum_{i=1}^k \alpha_i p(x_j|\mu_i, \Sigma_i) p(y_j|\Theta=i, x_j)) + \sum_{x_j \in D_u} \ln (\sum_{i=1}^k \alpha_i p(x_j|\mu_i, \Sigma_i))
$.
\textbf{求解 (EM算法)}

\subsection{半监督 SVM (S3VM / TSVM)}
\textbf{原理}:在最大化间隔的同时,尽量使超平面穿过数据低密度区域(TSVM)。\textbf{优化目标}:$
    \min_{\boldsymbol{w}, b, \hat{\boldsymbol{y}}, \boldsymbol{\xi}} \frac{1}{2} \|\boldsymbol{w}\|_2^2 + C_l \sum_{i=1}^l \xi_i + C_u \sum_{i=l+1}^m \xi_i$
s.t. 对 $D_l$ 满足标准约束;对 $D_u$ 满足 $\hat{y}_i(\boldsymbol{w}^T x_i + b) \geq 1 - \xi_i$。
\textbf{求解策略}:
1. 用 $D_l$ 训练初始 SVM。
2. 对 $D_u$ 进行预测指派伪标记 $\hat{y}$。
3. \textbf{迭代}:找出可能错误的伪标记样本($y_i \hat{y}_j < 0$ 且 $\xi > 0$),交换其标记,重新求解,逐步增大 $C_u$ 至 $C_l$。
\subsection{图半监督学习 (Graph-Based Methods)}
\textbf{构图}:节点为样本,边权重 $W_{ij}$ 为相似度(如高斯核)。
\textbf{能量函数}:$E(f) = \frac{1}{2} \sum_{i,j} W_{ij} (f(x_i) - f(x_j))^2 = \boldsymbol{f}^T (\boldsymbol{D} - \boldsymbol{W}) \boldsymbol{f}$。
\textbf{闭式解}:令 $\frac{\partial E}{\partial f_u} = 0$,得 $f_u = (D_{uu} - W_{uu})^{-1} W_{ul} f_l = (I - P_{uu})^{-1} P_{ul} f_l$。
\textbf{迭代式标记传播} (Label Propagation):
$
    \boldsymbol{F}(t+1) = \alpha \boldsymbol{S} \boldsymbol{F}(t) + (1-\alpha) \boldsymbol{Y}
$
其中 $\boldsymbol{S} = D^{-1/2} W D^{-1/2}$,收敛解为 $\boldsymbol{F}^* = (1-\alpha)(I - \alpha \boldsymbol{S})^{-1} \boldsymbol{Y}$。

\subsection{基于分歧的方法 (Disagreement-based Methods)}
\textbf{代表}:协同训练 (Co-training)。
\textbf{条件}:数据拥有两个充分且条件独立的视图 (View)。
\textbf{流程}:
1. 在两个视图上分别训练分类器 $h_1, h_2$。
2. 每个分类器挑选置信度最高的未标记样本(正/负),赋予伪标记并加入对方的训练集。
3. 迭代更新。

\subsection{半监督聚类 (Semi-Supervised Clustering)}
\textbf{监督信息类型}:
\begin{itemize}
    \item \textbf{约束}:必连 (Must-link, $M$) 与 勿连 (Cannot-link, $C$)。
    \item \textbf{标记样本}:少量样本已知所属簇。
\end{itemize}
\textbf{算法}:
1. \textbf{约束 k-means}:在分配簇时,若违反 $M$ 或 $C$ 约束,则不分配或分配给次优簇。2. \textbf{约束种子 k-means}:用有标记样本初始化簇中心 $\mu_j = \frac{1}{|S_j|} \sum_{x \in S_j} x$。
% ==========================================
% 第七部分:概率图模型
% ==========================================
\section{概率图模型}

\subsection{隐马尔可夫模型 (HMM)}
动态贝叶斯网,由隐藏状态序列 $y$ 和观测变量序列 $x$ 组成。
\begin{itemize}
    \item \textbf{三组参数} $\lambda = [A, B, \pi]$:
    \begin{itemize}
        \item 状态转移概率 $A$: $a_{ij} = P(y_{t+1}=s_j | y_t=s_i)$
        \item 输出观测概率 $B$: $b_{ij} = P(x_t=o_j | y_t=s_i)$
        \item 初始状态概率 $\pi$: $\pi_i = P(y_1=s_i)$
    \end{itemize}
    \item \textbf{齐次马尔可夫假设}:$P(y_t | y_{t-1}, \dots, y_1) = P(y_t | y_{t-1})$。
    \item \textbf{观测独立性假设}:$P(x_t | y_t, \dots) = P(x_t | y_t)$。
\end{itemize}

\subsection{马尔可夫随机场 (MRF)}
基于无向图,使用\textbf{团 (Clique)} 和\textbf{势函数 (Potential Function)} 定义联合概率。\textbf{联合概率分布}:
    $ P(\mathbf{x}) = \frac{1}{Z} \prod_{Q \in \mathcal{C}} \psi_Q(\mathbf{x}_Q) $
    其中 $\mathcal{C}$ 为极大团集合,$\psi_Q$ 为势函数 (通常 $\psi_Q \geq 0$),$Z = \sum_{\mathbf{x}} \prod_{Q \in \mathcal{C}} \psi_Q(\mathbf{x}_Q)$ 为规范化因子。 \textbf{马尔可夫性}:
    \begin{itemize}
        \item 全局:给定分离集 $x_C$,则 $x_A \perp x_B | x_C$。
        \item 局部/成对:非邻接节点在给定其他节点条件下独立。
    \end{itemize}

\subsection{条件随机场 (CRF)}
判别式无向图模型,对条件分布 $P(\mathbf{y}|\mathbf{x})$ 建模。\textbf{链式 CRF} (线性链):
    $ P(\mathbf{y}|\mathbf{x}) = \frac{1}{Z} \exp ( \sum_j \sum_i \lambda_j t_j(y_{i+1}, y_i, \mathbf{x}, i) + \sum_k \sum_i \mu_k s_k(y_i, \mathbf{x}, i) ) $
    其中 $t_j$ 为转移特征函数,$s_k$ 为状态特征函数。

\subsection{模型推断 (Inference)}
推断的核心是计算边际分布或条件分布。
\begin{itemize}
\item \hl{精确推断}:\begin{itemize}
    \item \textbf{变量消去法 (Variable Elimination)}:利用乘法对加法的分配律,将全局求和转化为局部求和,实质是动态规划。
    \item \textbf{信念传播 (Belief Propagation)}:通过节点间传递消息 $m_{ij}(x_j)$ 计算边际分布。节点 $x_i$ 的边际分布正比于接收消息的乘积.
\end{itemize}
\item \hl{近似推断}:\begin{itemize}
    \item \textbf{采样法}:通过构造平稳分布为目标分布 $p(\mathbf{x})$ 的马尔可夫链来产生样本。\textbf{Metropolis-Hastings (MH) 算法}:1.根据 $Q(\mathbf{x}^*|\mathbf{x}^{t-1})$ 采样候选 $\mathbf{x}^*$。
        2.计算接受率 $\alpha = \min \left(1, \frac{p(\mathbf{x}^*) Q(\mathbf{x}^{t-1}|\mathbf{x}^*)}{p(\mathbf{x}^{t-1}) Q(\mathbf{x}^*|\mathbf{x}^{t-1})} \right)$。
        3.以概率 $\alpha$ 接受 $\mathbf{x}^*$ 作为 $\mathbf{x}^t$。
    \textbf{吉布斯采样 (Gibbs Sampling)}:MH 的特例。每次固定其他变量,仅对一个变量 $x_i$ 根据 $p(x_i|\mathbf{x}_{\setminus i})$ 进行采样。
    \item \textbf{变分推断 (Variational Inference)}:使用简单分布 $q(\mathbf{z})$ 逼近复杂后验分布 $p(\mathbf{z}|\mathbf{x})$。\textbf{目标}:最小化 KL 散度 $KL(q \| p)$,等价于最大化证据下界 (ELBO)。
    $ \ln p(\mathbf{x}) = \mathcal{L}(q) + KL(q \| p) $
    \textbf{平均场理论}:假设 $q(\mathbf{z}) = \prod_{i} q_i(z_i)$,最优解满足 $\ln q_j^*(z_j) = \mathbb{E}_{i \neq j} [\ln p(\mathbf{x}, \mathbf{z})] + \text{const}$。
    \end{itemize}
\end{itemize}

% ==========================================
% 第八部分:规则学习
% ==========================================
\section{规则学习}
\textbf{序贯覆盖 (Sequential Coverage)}:逐条学习规则,覆盖正例并移除,直到覆盖所有正例。
\textbf{剪枝}:预剪枝(似然率)、后剪枝(REP, IREP, RIPPER)。
\textbf{一阶规则 (FOIL)}:使用一阶逻辑(谓词)。
FOIL增益:\mhl{Gain = \hat{m}_+ (\log_2 \frac{\hat{m}_+}{\hat{m}_+ + \hat{m}_-} - \log_2 \frac{m_+}{m_+ + m_-})}。$m_+$是标记正例中被规则判定为正例,$m_-$是标记负例中被规则判定为正例.\\
\textbf{冲突消解}:顺序规则、缺省规则、元规则。\\
\textbf{归纳逻辑程序设计 (ILP)}:
\begin{itemize}
    \item \textbf{最小一般泛化 (LGG)}:寻找覆盖两个例子的最特殊的一般规则。
    \item \textbf{逆归结}:演绎的逆过程。
    
    吸收:$\frac{p \leftarrow A \wedge B \quad q \leftarrow A}{p \leftarrow q \wedge B \quad q \leftarrow A}$辨识:$\frac{p \leftarrow A \wedge B \quad p \leftarrow A \wedge q}{q \leftarrow B \quad p \leftarrow A \wedge q}$\\内构:$\frac{p \leftarrow A \wedge B \quad p \leftarrow A \wedge C}{q \leftarrow B \quad p \leftarrow A \wedge q \quad q \leftarrow C}$互构:$\frac{p \leftarrow A \wedge B \quad q \leftarrow A \wedge C}{p \leftarrow r \wedge B \quad r \leftarrow A \quad q \leftarrow r \wedge C}$
\end{itemize}
% \includegraphics[width=0.3\textwidth]{image.png}

% ==========================================
% 第九部分:强化学习
% ==========================================
\section{强化学习 (RL)}
四元组 $<X, A, P, R>$。目标:最大化累积回报 $\mathbb{E}[\sum \gamma^t r_t]$。
\subsection{K-摇臂赌博机 (K-Armed Bandit)}
单步强化学习,最大化单步奖赏。
\begin{itemize}
    \item \textbf{$\epsilon$-贪心}: 以 $\epsilon$ 概率随机探索,以 $1-\epsilon$ 概率利用。
    \item \textbf{增量更新公式}: $Q_n(k) = Q_{n-1}(k) + \frac{1}{n} (v_n - Q_{n-1}(k))$。
    \item \textbf{Softmax}: 基于概率分布选择动作,概率 $P(k) = \frac{e^{Q(k)/\tau}}{\sum_{i} e^{Q(i)/\tau}}$ ($\tau$ 为温度参数)。
\end{itemize}
\subsection{值函数与Bellman方程}
\textbf{状态值函数}:$V^\pi(x) = \mathbb{E}_\pi [\sum \gamma^t r_t | x_0=x]$。
\textbf{动作值函数}:$Q^\pi(x,a) = \mathbb{E}_\pi [\sum \gamma^t r_t | x_0=x, a_0=a]$。\\
$ V^\pi(x) = \sum_{a} \pi(a|x) \sum_{x'} P(x'|x,a) [R(x,a,x') + \gamma V^\pi(x')] $\\
\textbf{最优Bellman方程}:
\mhl{ V^*(x) = \max_{a} Q^*(x,a) = \max_{a} \sum_{x'} P(x'|x,a) [R + \gamma V^*(x')] }\\
\mhl{ Q^*(x,a) = \sum_{x'} P(x'|x,a) [R + \gamma \max_{a'} Q^*(x', a')] }

\subsection{求解算法}
\begin{itemize}
    \item \textbf{动态规划 (Model-based)}:
    \begin{itemize}
        \item \textbf{策略迭代 (PI)}: 评估 $V^\pi$ $\to$ 改进 $\pi'(x) = \arg\max Q^\pi(x,a)$。
        \item \textbf{值迭代 (VI)}: 直接迭代 $V(x) \leftarrow \max_a \sum P(...)$。
    \end{itemize}
    \item \textbf{免模型 (Model-free)}:
    \begin{itemize}
        \item \textbf{蒙特卡罗 (MC)}: 采样轨迹,均值估计期望。\textbf{重要性采样}:$\rho_{t:T-1} = \prod_{k=t}^{T-1} \frac{\pi(A_k|S_k)}{b(A_k|S_k)}$。$V(S_t) \approx \rho_{t:T-1} G_t$
        \item \textbf{时序差分 (TD)}: 结合MC和DP。$V(x) \leftarrow V(x) + \alpha [r + \gamma V(x') - V(x)]$。
        \item \textbf{Q-Learning (Off-policy)}: $Q(x,a) \leftarrow Q(x,a) + \alpha [r + \gamma \max_{a'} Q(x',a') - Q(x,a)]$。
        \item \textbf{Sarsa (On-policy)}: $Q(x,a) \leftarrow Q(x,a) + \alpha [r + \gamma Q(x',a') - Q(x,a)]$。
    \end{itemize}
\end{itemize}
\textbf{探索与利用}:$\epsilon$-greedy (以 $\epsilon$ 概率随机)。

\subsection{值函数近似 (Value Function Approximation)}
针对状态空间巨大或连续的情况,使用参数化模型逼近值函数。
\begin{itemize}
    \item \textbf{线性近似}: $V_\theta(x) = \boldsymbol{\theta}^T \boldsymbol{\phi}(x)$,其中 $\boldsymbol{\phi}(x)$ 为特征向量。
    \item \textbf{梯度更新} (结合TD误差):
    $ \boldsymbol{\theta} \leftarrow \boldsymbol{\theta} + \alpha (r + \gamma V_\theta(x') - V_\theta(x)) \nabla_\theta V_\theta(x) $
    对于线性近似,$\nabla_\theta V_\theta(x) = \boldsymbol{\phi}(x)$。
\end{itemize}

\subsection{模仿学习 (Imitation Learning)}
利用专家示范数据 $\mathcal{D} = \{\tau_1, \tau_2, ...\}$ 进行学习。
\begin{itemize}
    \item \textbf{直接模仿 (Behavior Cloning)}: 将专家轨迹视为分类/回归训练集,直接学习策略 $\pi: X \to A$。
    \item \textbf{逆强化学习 (IRL)}: 假设专家策略是最优的,从数据中反推奖赏函数 $R$。
    目标:寻找$R^*$使得$\mathbb{E}_{\pi^*} [\sum \gamma^t R^*(s_t)] \geq \mathbb{E}_{\pi} [\sum \gamma^t R^*(s_t)]$
    得到 $R^*$ 后,再通过标准RL算法求解最优策略。
\end{itemize}
\end{multicols*}
\end{document}